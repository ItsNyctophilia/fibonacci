\documentclass{article}

\title{Fibonacci}
\author{Dakota Kellogg, James Viner, Roberthon Meyers}
\date{}

% Referenced https://stackoverflow.com/a/21344989
\newcommand{\code}[1]{\textcolor{SkyBlue}{\texttt{#1}}}
\newcommand{\co}[1]{\textcolor{red}{\texttt{#1}}}

\usepackage[dvipsnames]{xcolor}
% Referenced https://tex.stackexchange.com/a/351969
\usepackage{fancyvrb}

\begin{document}

\maketitle

\section*{Project Summary}
The task is to write a program in x86 assembly language that uses the first command-line parameter as a number N from 0 through 100. The program should print out F(N), in hexadecimal. 

\section*{Features Targeted}

\subsubsection*{Design Plan, Test Plan, and Write up using \LaTeX}
Write the write up, test plan, and design plan using \TeX~or \LaTeX~language
(be sure to include the source files).

\subsubsection*{Man page}
Write a \co{man(1)} page to document the program.

\subsubsection*{Octal output}
Add a command-line option \code{-o} for octal output.  

\subsubsection*{Decimal output}
add a command-line option \code{-d} for decimal output up to f(100)/f(300)

\pagebreak
\section*{Architecture}
The code consists of three main parts, processing of options, calculation of number, and printing of the number.

\section*{User Interface}
\noindent
Uses the CLI. The user can enter options that affect the behavior of the
program. The user must enter a mandatory command during program invocation. 

\section*{Approach}
\begin{enumerate}
	\item Review Fibonacci sequence 
	\item Sketch out command line input portion
	\item Sketch out Fibonacci calculation portion
        \item Sketch out printing portion 
        \item Implement options for extra features 
        \item Review code for extra features
        \item Implement translation between binary to octal/decimal
	\item Review completed code with group
	\item Implement discussed changes
	\item Create post-project write-ups
\end{enumerate}

\end{document}
