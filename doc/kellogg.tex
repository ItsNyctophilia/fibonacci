\documentclass{article}

\begin{document}

\title{Post project write-up - Fibonacci}

\author{Dakota Kellogg}

\maketitle

\section{Introduction}
The task was to write a program in x86 assembly language that uses the first command-line parameter as a number N from 0 through 100. The program should print out F(N), in hexadecimal. 


\section{Option Handling}
One of the primary challenges encountered during the project was proper option handling. It is easy to handle CLI in C, or Python, but not as simple as I thought it would be in Assembly. 

\section{Data Type Conversion}
Another challenge faced during the project was incorrect data type conversions. Specifically, being able to get the math register to go over 128 bits. It was a long time trying to figure that one out

\section{Conclusion}
In conclusion, the project faced several challenges, such as option handling, and data type conversions. However, these issues were eventually resolved with debugging, persistence, and proper troubleshooting. 
\end{document}
